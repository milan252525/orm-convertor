\section{ORM selection process}
Now that we know the basic concepts of ORMs have been introduced, we can move on to the criteria for selecting specific frameworks for our evaluation.

Important factors will be popularity and community support. Those can be estimated from the number of downloads on the .NET package manager NuGet\footnote{\url{https://www.nuget.org/}} or activity on GitHub\footnote{\url{https://github.com/}}. Since not all software is open source, we will limit ourselves to frameworks that are. We also  exclude outdated ORMs. Selected ones must have received at least some form of update or security fix in recent years.

Support for recent .NET versions is another essential factor. The versioning history of .NET is somewhat complicated. The original implementation, .NET Framework, was Windows-only and went up to version 4.8.1. Around 2016, Microsoft performed a major rework, focusing on cross-platform implementation. With a new name .NET Core, they started again at version 1.0. After .NET Core 3.1, the "Core" naming was dropped, and versions continued as .NET 5, .NET 6 and so on. Today, developing new applications using .NET Framework is not recommended, as it only receives security patches. The current long term support version is .NET 8, while .NET 9 is a standard term support release. In this text, we refer to the legacy platform as ".NET Framework" and to the new one as ".NET Core" or simply ".NET". \cite{NETFrameworkVersions}\cite{NETversions}

Another version to be aware of is .NET Standard. It is a formal API specification designed to unify .NET implementations. It allows libraries to be compatible with both .NET Framework and Core by targeting a common subset of APIs. However, this can restrict access to newer features and optimizations. \cite{NETStandard}

This work will be limited to ORMs supporting Microsoft SQL Server. As we will see, this is not a restrictive requirement. All major .NET ORMs offer SQL support.

Our goal is to select a representative and diverse sample for comparison and testing. We aim to include at least two of micro and macro ORM frameworks. We will also add frameworks with notably interesting capabilities.

\autoref{tab:orm-docs} lists all notable considered .NET ORM frameworks with a link to their homepage or repository.

\begin{table}[h]
\definecolor{lightgreen}{RGB}{217, 234, 211}
\centering
\begin{tabular}{|l|l|}
\hline
\textbf{ORM} & \textbf{URL} \\
\hline
\cellcolor{lightgreen}Dapper & \url{https://github.com/DapperLib/Dapper} \\
\cellcolor{lightgreen}Entity Framework 6 & \url{https://learn.microsoft.com/en-us/ef/ef6/} \\
\cellcolor{lightgreen}Entity Framework Core & \url{https://learn.microsoft.com/en-us/ef/core/} \\
\cellcolor{lightgreen}LINQ to DB (linq2db) & \url{https://linq2db.github.io/} \\
Massive & \url{https://github.com/FransBouma/Massive} \\
Mighty & \url{https://github.com/MightyOrm/Mighty} \\
\cellcolor{lightgreen}NHibernate & \url{https://nhibernate.info/} \\
Norm.NET & \url{https://vb-consulting.github.io/norm.net/} \\
\cellcolor{lightgreen}PetaPoco & \url{https://github.com/CollaboratingPlatypus/PetaPoco} \\
\cellcolor{lightgreen}RepoDB & \url{https://repodb.net/} \\
SqlMarshal & \url{https://github.com/kant2002/SqlMarshal} \\
XPO & \url{https://devexpress.com/Products/NET/ORM/} \\
\hline
\end{tabular}
\caption{Considered ORMs with selected ones highlighted in green\label{tab:orm-docs}}
\end{table}

XPO was not considered due to not being open source. The other excluded frameworks were found to be outdated or inactive. This was primarily determined by the lack of recent GitHub commits and low community engagement, such as star count. 

The final selection includes seven ORMs. The obvious choices were \textbf{Dapper}, \textbf{NHibernate} and \textbf{Entity Framework Core}, as they are the most widely used. We also include \textbf{Entity Framework 6} for completeness --- despite being older, it remains widely spread among legacy projects and meaningfully differs from EF Core, particularly in version compatibility. For the micro category, we selected \textbf{PetaPoco} to oppose Dapper. During our research, we also encountered frameworks that fall somewhere between micro and macro, offering some abstraction without the full complexity of macro ORMs. \textbf{LINQ to DB} extensively supports LINQ but stays lightweight. And \textbf{RepoDB} promises to be "easiest-to-use"\cite{RepoDB}. We will look into each ORM in detail and explore its capabilities in the next section.