\chapter{Related work}\label{chapter:related_work}
Including an \acrshort{orm} into an application with database access can significantly reduce the amount of code written and, in certain cases, improve access time~\cite{optimizatio_code_lines_time}. However, such speedup depends on selecting an appropriate framework and deep domain knowledge. Multiple analyses have been conducted in the past to guide this selection.

A performance evaluation over a PostgreSQL database~\cite{orm_comp_postgre} identified Dapper as the most efficient framework in terms of resource usage, outperforming both \acrshort{ef} Core and NHibernate. Nevertheless, the authors also noted Dapper's limited feature set, suggesting that \acrshort{ef} Core would be more suitable when a full-featured \acrshort{orm} is necessary. The findings align with the conclusions reached in our comparison.

Another study~\cite{myllyahoforsberg2022orm} assessed Dapper and \acrshort{ef} Core over \acrshort{mssql}. This comparison not only analyzed resource allocation but also considered aspects of code complexity, such as cyclomatic complexity\footnote{\url{https://learn.microsoft.com/en-us/visualstudio/code-quality/code-metrics-cyclomatic-complexity}} and lines of code. The authors concluded by recommending Dapper for performance and \acrshort{ef} Core for development efficiency, aligning with our findings.

An older comparison~\cite{orm_comp_2020} examined the same three frameworks over \acrshort{mssql} but did not reach a definitive conclusion. Instead, the authors advised selecting different frameworks based on specific operation types. However, since their analysis was based on framework versions from approximately 2018, rapid advancements, especially within \acrshort{ef} Core, limit its relevance today. Furthermore, the use of default configurations, including caching and lazy loading without consistent alignment across frameworks, could have influenced the outcome.

To the best of our knowledge, no other recent research has performed a comparison of .NET \acrshort{orm}(s) on the scale presented in this work. Specifically, PetaPoco, RepoDB, and LINQ To DB have not been previously evaluated. In particular, a systematic evaluation of more than twenty read queries across seven frameworks has not been previously conducted.

Even after selecting an optimal \acrshort{orm}, performance limitations remain a known and longstanding issue~\cite{survey_hule_ranawat_2023}. Various methods for addressing these limitations have been explored, including parallelization~\cite{orm_parallel}, optimizing \acrshort{orm} configuration through multi-objective genetic algorithms~\cite{orm_configuration_optimization} or adapting queries by automatically selecting necessary columns and applying join elimination~\cite{quartarone2020adaptive}. Another performance enhancement strategy involves predicting future queries and prefetching results~\cite{prefetching}.
% případně lze vyvinout vlastní ORM - https://www.sciencedirect.com/science/article/pii/S2405896319327107

Combining multiple \acrshort{orm}(s) within a single application presents an alternative approach. However, to the best of our knowledge, no tool currently exists to assist with optimal framework selection. To address this gap, our work proposes an automated advisor to select the most suitable frameworks based on given constraints. 

Improved performance might also be achieved through modifications to the database schema. Certain data may not be suitable for relational databases, making a multi-model schema potentially a better solution for diverse data. 

To address the complexities associated with managing multi-model databases and ensuring consistency, a set of tools has been developed. MM-evocat~\cite{mm_evocat} supports schema evolution over multi-model databases using an abstract representation based on category theory. MM-quecat~\cite{mm_quecat} leverages this representation to provide a unified query language over multi-model data, while MM-evoque~\cite{mm_evoque} ensures that schema changes are propagated across all database queries. 

When it comes to translation between different \acrshort{orm}(s), an approach using artificial intelligence could be investigated. The area has seen recent progress in improved code-writing capabilities~\cite{evaluatingllms} of \acrlong{llm}s. \acrshort{llm}s\cite{transagentllm} and unsupervised machine translation~\cite{translation_unsupervised} have been used for translation across different programming languages.% in addition to .

Deep learning methods have also been extensively studied~\cite{DBLP:journals/vldb/KatsogiannisMeimarakisK23} for translating natural language into \acrshort{sql} queries. Tools such as IRNet~\cite{text-to-sql-irnet} or RAT-SQL~\cite{text-to-sql-rat} demonstrate capability in generating queries directly from user input.


% kdyby bylo potřeba prodloužit ještě jsem našel:
% https://arxiv.org/abs/2403.09060 - Query Rewriting via Large Language Models
% https://arxiv.org/abs/2401.02115 - Using LLM to select the right SQL Query from candidate