\section{Important concepts}
To better understand \acrshort{orm}(s) and the .NET\footnote{\url{https://dotnet.microsoft.com/en-us/}} ecosystem, important concepts have to be examined first, starting with relational data management in .NET applications. 

To request data, \acrshort{sql} query must be composed, executed through an open database connection, and results must be manually parsed. This can be achieved with the help of the standard .NET library, particularly ADO.NET.\footnote{\url{https://learn.microsoft.com/en-gb/dotnet/framework/data/adonet/}} However, by directly using \acrshort{sql} in raw strings, dependencies on a query language that lacks static typing in the context of C\# are introduced. Based on our observations, this can potentially lead to additional complexity. Nevertheless, ADO.NET remains crucial, and most .NET applications have it as a hidden dependency. All of the \acrshort{orm} frameworks introduced later rely on it for database communication. 

To lower complexity related to database communication and data persistence, abstraction provided by \acrshort{orm}(s) is used. The level of abstraction varies depending on the specific framework. There are generally two categories, micro and macro \acrshort{orm}(s)~\cite{shapovalov2024micro}. On one end of the spectrum, there are lightweight, usually performance-oriented frameworks. They offer a narrow set of features, some simple abstraction over ADO.NET. Some of those might not even be full \acrshort{orm}(s), omitting the ``relational'' part entirely~\cite{Dapper}. On the opposite end appear full-fledged frameworks that completely abstract away database communication, providing abstraction over \acrshort{sql} in the form of different query languages and an extensive set of features to simplify the work of developers.

\subsection{ADO.NET}
ADO.NET\footnote{\url{https://learn.microsoft.com/en-us/dotnet/framework/data/adonet/ado-net-overview}} provides access to different data sources, not limited to only databases. It provides components and models for connecting to the sources, executing commands, and retrieving results. All this functionality is built into \texttt{System.Data} library.

\begin{example}
\small
A shortened example from Microsoft documentation\footnote{\url{https://learn.microsoft.com/en-us/dotnet/framework/data/adonet/ado-net-overview}} demonstrates the usage of key ADO.NET objects, namely \texttt{SqlConnection}, \texttt{SqlCommand} and \texttt{SqlDataReader}. Managing the connection, building the query command, and reading results introduces a fair amount of overhead and manual control. The indexed access to \texttt{SqlDataReader} inside the \texttt{while} loop can end up being error-prone and increase the risk of runtime exceptions. Although this approach is functional, it highlights why an abstraction would be more desirable.
\qed

\begin{lstlisting}[language=CSharp]
const string connectionString = "...";

const string queryString =
    "SELECT ProductID, UnitPrice, ProductName from dbo.products "
    + "WHERE UnitPrice > @pricePoint "
    + "ORDER BY UnitPrice DESC;";

const int paramValue = 5;

using (SqlConnection connection = new(connectionString))
{
    SqlCommand command = new(queryString, connection);
    command.Parameters.AddWithValue("@pricePoint", paramValue);

    connection.Open();
    SqlDataReader reader = command.ExecuteReader();
    while (reader.Read())
    {
        Console.WriteLine($"{reader[0]}\t{reader[1]} ...");
    }
    reader.Close();
}
\end{lstlisting}
\end{example}

\subsection{Entities and mapping}
Entities are a set of classes either mirroring database tables or reflecting a domain model. Entities should be plain old \acrshort{clr} objects (\acrshort{poco}(s)). The term is derived from plain old Java objects (\acrshort{pojo}(s)). It refers to classes that do not inherit from any base class or interface. And they are not dependent on any library~\cite{Fowler2003POJO}.

\begin{example}
\small
An example of a purchase order class demonstrates a composition of \acrshort{poco}. It contains a set of properties with basic .NET data types. It has no dependencies and can be used within any framework. This entity contains no rules for mapping. It would work only with a query whose result columns match the entity properties. And of course, data types and nullability would have to match as well.
\qed
\begin{lstlisting}[language=CSharp]
class PurchaseOrder
{
    int PurchaseOrderID { get; set; }
    int SupplierID { get; set; }
    DateTime OrderDate { get; set; }
    DateTime? ExpectedDeliveryDate { get; set; }
    string? Comments { get; set; }
}
\end{lstlisting}
\end{example}

Micro \acrshort{orm}(s) do not usually use any entity mapping. They simply load resulting records into entities that match projected columns. Frameworks performing more complex non-read operations require further configuration to avoid mismatches. In some use cases, such as domain-driven development, entities might be initially designed in an application and then the corresponding schema generated with the help of some tool~\cite{FowlerDDD}.

Entity mapping might be done in multiple ways. The simplest form is inferred mapping or global conventions. The framework guesses, e.g., table and column names, data types, and nullability from entity declaration or global rules. An example of such a framework is Entity Framework Core\footnote{\url{https://learn.microsoft.com/en-us/ef/core/modeling/}}, which, without any mapping, uses default conventions to infer the mapping. Clearly, inferred mapping lacks constraints including concrete data types, numeric precision, or encoding. Those details can be specified by property and class attributes. Coupling the mapping with the entity makes it framework-dependent, which can be problematic if the domain model is meant to remain isolated from database access to avoid abstraction leakage. Mapping through code or a configuration file allows separation and is usually more expressive and powerful.

\begin{example}
\small
Let us consider two different mapping approaches and compare them. In the first instance, \texttt{PurchaseOrder} entity is mapped in Entity Framework Core\footnote{\url{https://learn.microsoft.com/en-gb/ef/core/}} using attributes and the rest of the mapping is inferred. For example, \texttt{OrderDate} will likely be mapped to a \texttt{DATETIME2} column type, with default precision. Of course, the inference is optional and more configuration attributes or mapping by code can be added to avoid any uncertainties. 
\begin{lstlisting}[language=CSharp]
[Table("PurchaseOrders", Schema = "Purchasing")]
class PurchaseOrder
{
    [Key]
    int PurchaseOrderID { get; set; }

    int SupplierID { get; set; }

    DateTime OrderDate { get; set; }
}
\end{lstlisting}
The second mapping is in the form of \acrshort{xml} mapping file using NHibernate \acrshort{orm}\footnote{\url{https://nhibernate.info/}}. This assumes a corresponding \acrshort{poco} is somewhere else in the project. This configuration is more verbose and requires more information, but prevents any issues with inference. The mapping isolation from the entity might be good for architecture, but in this case, it completely prevents automated refactoring and schema change reflection.

\qed
\begin{lstlisting}[language=CSharp]
<?xml version="1.0" encoding="utf-8" ?>
<hibernate-mapping xmlns="urn:nhibernate-mapping-2.2" namespace="NHibernateEntities">
    <class name="NHibernateEntities.PurchaseOrder, NHibernateEntities" table="PurchaseOrders" schema="Purchasing">
        <id name="PurchaseOrderID" column="PurchaseOrderID" type="int">
            <generator class="identity" />
        </id>
        <property name="SupplierID" not-null="true" />
        <property name="OrderDate" not-null="true" />
    </class>
</hibernate-mapping>
\end{lstlisting}
\end{example}

Relationships are usually mapped through navigation properties. If the relationship has a single target, the navigation property is of the related entity's type. In contrast, if the relationship can target multiple entries, a collection of the related type (e.g. \texttt{List<T>}) is used instead.

\subsection{Querying}\label{sec:queries}
Relational databases generally support only \acrshort{sql} query language. Some \acrshort{orm}(s) use \acrshort{sql} directly. But relational data and queries do not translate well into a statically typed object-oriented language. That is why Language Integrated Queries (\acrshort{linq})\footnote{\url{https://learn.microsoft.com/en-us/dotnet/csharp/linq/}} were eventually introduced into C\#.\footnote{\url{https://learn.microsoft.com/en-gb/dotnet/csharp/}} They abstract the data source and can operate over relational databases as well as other sources such as \acrshort{xml} documents, object databases, or web services.

There are two ways to write \acrshort{linq} queries. The first is called \textit{declarative query syntax} and the second \textit{method syntax}. The query syntax is more \acrshort{sql}-like and should be easier to read. However, it does not syntactically fit within the C\# language, and the compiler translates it into the method syntax anyway.

\begin{example}
\small
To get an idea of how the two approaches differ, let us consider an example adapted from the documentation of \acrshort{linq}.\footnote{\url{https://learn.microsoft.com/en-us/dotnet/csharp/linq/get-started/introduction-to-linq-queries}} The two queries execute the same way and return identical results. In fact, they get translated into identical lower-level code by the compiler.

\begin{lstlisting}[language=CSharp]
int[] numbers = [ 5, 10, 8, 3, 6, 12 ];

//Query syntax:
IEnumerable<int> numQuery1 =
    from num in numbers
    where num % 2 == 0
    orderby num
    select num;

//Method syntax:
IEnumerable<int> numQuery2 = numbers
    .Where(num => num % 2 == 0)
    .OrderBy(n => n);
\end{lstlisting}
\qed
\end{example}

By applying different methods, operations like filtering, sorting, or grouping are applied on the data source. However, no part of the query is executed until iteration over the result occurs. Or until a terminal method such as \texttt{ToList}, \texttt{First}, \texttt{Single}, or \texttt{Count} is invoked. 

If the source operates on in-memory data, it should implement \texttt{IEnumerable} interface. For remote data, such as those stored in a database, \texttt{IQueryable} interface must be implemented. The source can translate the method calls into any query language. In the case of \acrshort{orm}(s), it will likely translate \acrshort{linq} method calls into \acrshort{sql} statements.\footnote{\url{https://learn.microsoft.com/en-us/dotnet/csharp/linq/get-started/introduction-to-linq-queries}}