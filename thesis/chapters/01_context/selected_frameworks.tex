\section{Selection process}
With the basic concepts of \acrshort{orm}(s) established, the criteria for selecting specific frameworks for evaluation can now be introduced, potential frameworks reviewed, and a representative subset ultimately selected for detailed analysis.

Important factors are popularity and community support. Those can be estimated from the number of downloads on the .NET package manager NuGet\footnote{\url{https://www.nuget.org/}} or activity on GitHub.\footnote{\url{https://github.com/}} As not all software is open source, the scope is limited to frameworks that are. As not all software is open source, the scope is limited to frameworks that are. Outdated \acrshort{orm}(s) are also excluded. To be considered, a framework must have received at least some form of update or security fix in recent years.

Support for recent .NET versions is another essential factor. The versioning history of .NET is somewhat complicated. The original implementation, .NET Framework, was Windows-only and went up to version 4.8.1. Around 2016, Microsoft performed a major rework, focusing on cross-platform implementation. With a new name .NET Core, they started again at version 1.0. After .NET Core 3.1, the "Core" naming was dropped, and versions continued as .NET 5, .NET 6 and so on. Today, developing new applications using .NET Framework is not recommended, as it only receives security patches. The current long term support version is .NET 8, while .NET 9 is a standard term support release. In this text, we refer to the legacy platform as ``.NET Framework'' and to the new one as ``.NET Core'' or simply ``.NET''.\footnote{\url{https://dotnet.microsoft.com/en-us/platform/support/policy/dotnet-core}, \url{https://dotnet.microsoft.com/en-us/learn/dotnet/what-is-dotnet-framework}}

Another version to be aware of is .NET Standard. It is a formal \acrshort{api} specification designed to unify .NET implementations. It allows libraries to be compatible with both .NET Framework and Core by targeting a common subset of \acrshort{api}(s). However, this can restrict access to newer features and optimizations.\footnote{\url{https://dotnet.microsoft.com/en-us/platform/dotnet-standard}}

This work focuses on \acrshort{orm}(s) that support Microsoft SQL Server (\acrshort{mssql}). This is not a restrictive requirement, as all major .NET \acrshort{orm}(s) provide support for \acrshort{mssql}.

The goal is to select a representative and diverse sample of \acrshort{orm} frameworks for comparison and testing. Table~\ref{tab:orm-docs} presents a selection of considered .NET \acrshort{orm} frameworks along with links to their homepages or repositories. The final selection aims to include at least two frameworks representing both micro and macro \acrshort{orm} types, as well as frameworks that offer particularly notable capabilities.

\begin{table}[ht!]
\footnotesize
\def\arraystretch{1.25}
\definecolor{lightgreen}{RGB}{217, 234, 211}
\centering
\begin{tabular}{ll}
\toprule
\textbf{ORM} & \textbf{URL} \\
\midrule
\cellcolor{lightgreen}Dapper & \url{https://github.com/DapperLib/Dapper} \\
\cellcolor{lightgreen}Entity Framework 6 & \url{https://learn.microsoft.com/en-us/ef/ef6/} \\
\cellcolor{lightgreen}Entity Framework Core & \url{https://learn.microsoft.com/en-us/ef/core/} \\
\cellcolor{lightgreen}LINQ to DB (linq2db) & \url{https://linq2db.github.io/} \\
Massive & \url{https://github.com/FransBouma/Massive} \\
Mighty & \url{https://github.com/MightyOrm/Mighty} \\
\cellcolor{lightgreen}NHibernate & \url{https://nhibernate.info/} \\
Norm.NET & \url{https://vb-consulting.github.io/norm.net/} \\
\cellcolor{lightgreen}PetaPoco & \url{https://github.com/CollaboratingPlatypus/PetaPoco} \\
\cellcolor{lightgreen}RepoDB & \url{https://repodb.net/} \\
SqlMarshal & \url{https://github.com/kant2002/SqlMarshal} \\
XPO & \url{https://devexpress.com/Products/NET/ORM/} \\
\bottomrule
\end{tabular}
\caption{Considered ORMs with selected ones highlighted in green\label{tab:orm-docs}}
\end{table}

From the list of \acrshort{orm} frameworks, XPO was not considered due to not being open source. The other excluded frameworks were found to be outdated or inactive. This was primarily determined by the lack of recent GitHub commits and low community engagement, such as star count. 

The final selection (highlighted in light green) includes seven \acrshort{orm}(s).\textbf{Dapper}, \textbf{NHibernate}, and \textbf{Entity Framework Core} are chosen as the most widely used options. \textbf{Entity Framework 6} is included for completeness -- despite being older, it remains widely spread among legacy projects and meaningfully differs from \acrshort{ef} Core, particularly in terms of version compatibility. \textbf{PetaPoco} is selected to contrast Dapper in the micro category. During the research process, frameworks that fall between micro and macro categories were also identified, offering some abstraction without the full complexity of macro \acrshort{orm}s. \textbf{LINQ to DB} extensively supports \acrshort{linq} but stays lightweight. \textbf{RepoDB} is included based on its claim of being the ``easiest-to-use''~\cite{RepoDB}. Each \acrshort{orm} is examined in detail in the following section.