\chapter*{Conclusion}
\addcontentsline{toc}{chapter}{Conclusion}

This thesis addresses the problem of adapting application code under evolving requirements, particularly in the area of Object-Relational Mapping (\acrshort{orm}). All initial goals were achieved:

\begin{itemize}
    \item A comparative analysis of seven .NET \acrshort{orm} frameworks highlighted their feature sets and evaluated their performance using a set of representative queries.
    \item A unified intermediate representation was introduced for entity classes, mapping configurations, and queries, forming the basis for cross-framework translation.
    \item A set of interfaces and algorithms was designed to enable translation and migration of entity and repository classes between frameworks.
    \item An optimization advisor module was proposed and implemented, formulating the framework selection task as an Integer Linear Programming problem. The module leverages empirical performance data to recommend the optimal \acrshort{orm} framework(s) for a given queries and constraints.
    \item A prototype translation tool was implemented, demonstrating the feasibility of the approach and supporting Dapper, NHibernate, and Entity Framework Core.
\end{itemize}

In future work, the performance analysis could be extended to cover more types of operations beyond read queries. Additional statistics could be measured besides runtime and memory allocation. 

The implementation of translation could be expanded to support more \acrshort{orm}(s) and even additional programming languages. To further improve usability, a command-line tool could be developed alongside to enable application-wide translation, automatically adapting configuration and data flow, rather than requiring users to copy code fragments via the web interface. A visual designer could be implemented on top of the tool to allow intuitive schema management independent of any \acrshort{orm} framework.

Additionally, the advisor could be enhanced to handle more types of constraints and optimizations beyond application code. Queries could be executed across multiple database systems, and a migration to a more suitable one could be recommended. Schema evolution could be supported by generated migration scripts and adaptation of existing mapping configuration. Future versions could also incorporate automated detection of potential security or performance issues arising from translated queries or generated code.
