%%% Please fill in basic information on your thesis, which will be automatically
%%% inserted at the right places. You need to replace \xxx{...} by real data.

% Type of your thesis:
%	"bc" for Bachelor's
%	"mgr" for Master's
%	"phd" for PhD
%	"rig" for rigorosum
\def\ThesisType{mgr}

% Language of your study programme:
%	"cs" for Czech
%	"en" for English
\def\StudyLanguage{cs}

% Thesis title in English (exactly as in the official assignment)
% (Note: \xxx is a "ToDo label" which makes the unfilled visible. Remove it.)
\def\ThesisTitle{Framework-Agnostic Query Adaptation: Ensuring SQL Compatibility Across .NET Database Frameworks}

% Author of the thesis (you)
\def\ThesisAuthor{Bc. Milan Abrahám}

% Year when the thesis is submitted
\def\YearSubmitted{2025}

% Name of the department or institute, where the work was officially assigned
% (according to the Organizational Structure of MFF UK in English,
% see https://www.mff.cuni.cz/en/faculty/organizational-structure,
% or a full name of a department outside MFF)
\def\Department{Department of Software Engineering}

% Is it a department (katedra), or an institute (ústav)?
\def\DeptType{Department}

% Thesis supervisor: name, surname and titles
\def\Supervisor{Ing. Pavel Koupil, Ph.D.}

% Supervisor's department (again according to Organizational structure of MFF)
\def\SupervisorsDepartment{Department of Software Engineering}

% Study programme (does not apply to rigorosum theses)
\def\StudyProgramme{Computer Science - Software and Data Engineering}

% An optional dedication: you can thank whomever you wish (your supervisor,
% consultant, who provided you with tea and pizza, etc.)
\def\Dedication{%
I would like to thank my supervisor, Pavel Koupil, for his guidance throughout the entire process and for his insightful advice. His genuine interest in the topic and contributions helped shape the direction of this thesis. I am also grateful to my family for their encouragement and support. Finally, I would like to thank my colleagues at work for giving me the time and valuable advice related to the topic.
}

% Abstract (recommended length around 80-200 words; this is not a copy of your thesis assignment!)
\long\def\Abstract{%
Modern software systems face rapidly evolving requirements that impact both their underlying data and executed queries. While automatic adaptation on the database side has received considerable attention, there is a significant gap regarding how such changes affect application code, particularly in the context of Object-Relational Mapping (ORM).

In this thesis, we introduce a novel approach for the translation of ORM configurations and queries. By analysing seven .NET ORM frameworks, we propose a unified representation and a set of algorithms enabling automatic translation. Building on this foundation, we present an optimization advisor module. This advisor systematically translates and executes queries across all target frameworks, collecting empirical performance data. Using Integer Linear Programming techniques, it recommends the optimal combination of frameworks for a given set of queries and constraints.
}

\def\AbstractXMP{%
Modern software systems face rapidly evolving requirements that impact both their underlying data and executed queries. While automatic adaptation on the database side has received considerable attention, there is a significant gap regarding how such changes affect application code, particularly in the context of Object-Relational Mapping (ORM).
In this thesis, we introduce a novel approach for the translation of ORM configurations and queries. By analysing seven .NET ORM frameworks, we propose a unified representation and a set of algorithms enabling automatic translation. Building on this foundation, we present an optimization advisor module. This advisor systematically translates and executes queries across all target frameworks, collecting empirical performance data. Using Integer Linear Programming techniques, it recommends the optimal combination of frameworks for a given set of queries and constraints.
}

% 3 to 5 keywords (recommended) separated by \sep
% Keywords are useful for indexing and searching for the theses by topic.
\def\ThesisKeywords{
ORM comparison\sep ORM translation\sep optimization\sep query performance
}

% If any of your metadata strings contains TeX macros, you need to provide
% a plain-text version for use in XMP metadata embedded in the output PDF file.
% If you are not sure, check the generated thesis.xmpdata file.
\def\ThesisAuthorXMP{\ThesisAuthor}
\def\ThesisTitleXMP{\ThesisTitle}
\def\ThesisKeywordsXMP{\ThesisKeywords}
%\def\AbstractXMP{\Abstract}

% If your abstracts are long and do not fit in the infopage, you can make the
% fonts a bit smaller by this setting. (Also, you should try to compress your abstract more.)
%\def\InfoPageFont{}
\def\InfoPageFont{\small}  % uncomment to decrease font size

% If you are studing in a Czech programme, you also need to provide metadata in Czech:
% (in English programmes, this is not used anywhere)

\def\ThesisTitleCS{Zajištění kompatibility SQL dotazů napříč různými databázovými frameworky v .NET}
\def\DepartmentCS{Katedra softwarového inženýrství}
\def\DeptTypeCS{Katedra}
\def\SupervisorsDepartmentCS{Katedra softwarového inženýrství}
\def\StudyProgrammeCS{Informatika - Softwarové a datové inženýrství}

\def\ThesisKeywordsCS{%
porovnání ORM\sep překlad ORM\sep optimalizace\sep výkon dotazování
}

\long\def\AbstractCS{%
Moderní softwarové systémy se musí přizpůsobovat stále se měnícím požadavkům, které ovlivňují nejen data, ale i vykonávané dotazy. Zatímco automatické úpravy na úrovni databáze jsou dobře prozkoumané, dopad těchto změn na aplikační kód, zejména v kontextu objektově-relačního mapování (ORM), zůstává téměř nezmapovaný. V této práci proto představujeme nový přístup k automatickému překladu ORM konfigurací a dotazů.

Na základě analýzy sedmi .NET ORM frameworků představujeme sjednocující abstrakci a sadu algoritmů umožnující automatický překlad. Nad tímto základem stavíme modul optimalizačního poradce, který dotazy systematicky přeloží a spustí ve vybraných cílových frameworcích. Po shromáždění emprických data o výkonu a formulaci problému jako instance celočíselného lineárního programování doporučí optimální kombinaci frameworků pro danou sadu dotazů a omezení.
}
